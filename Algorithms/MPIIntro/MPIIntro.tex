\lab{Algorithms}{An Introduction to Parallel Programming using MPI}{Introduction to Parallel Programming}
\objective{Learn the basics of parallel computing on distributed memory machines using MPI for Python}
\label{lab:MPI_Intro}

\section*{Why Parallel Computing?}
Over the past few decades, vast increases in computational power have come through 
increased single processor performance, which have almost wholly been driven by smaller transistors.
However, these faster transistors generate more heat, which destabilizes circuits. 
The so-called ``heat wall'' refers to the physical limits of single processors to 
become faster because of the growing inability to sufficiently dissipate heat.

To get around this physical limitation, parallel computing was born.
It began, as with most things, in many different systems and styles, involving various systems. 
Some systems had shared memory between many processors and some systems had a separate 
program for each processor while others ran all the processors off the same base program.

Today, the most commonly used method for high performance computing is a single-program,
multiple-data system with message passing. 
Essentially, these supercomputers are made up of many, many normal computers, each with their own memory. 
These normal computers are all running the same program and are able to communicate with each other. Although they all run the same program file, each process takes a different execution path through the code, as a result of the interactions that message passing makes possible.

Note that the commands being executed by process \emph{a} are going to be different 
from process \emph{b}. Thus, we could actually write two separate computer programs 
in different files to accomplish this. 
However, the most common method today has become to write both processes in a single program, 
using conditionals and techniques from the Message Passing Interface to control which 
lines of execution get assigned to which process.

This is a very different architecture than ``normal'' computers, and it requires a 
different kind of software. You can't take a traditional program and expect it to 
magically run faster on a supercomputer; you have to completely design a new algorithm 
to take advantage of the unique opportunities parallel computing gives.

\section*{MPI: the Message Passing Interface}
At its most basic, the Message Passing Interface (MPI) provides functions for sending 
and receiving messages between different processes.

MPI was developed out of the need for standardization of programming parallel systems. 
It is different than other approaches in that MPI does not specify a particular language. 
Rather, MPI specifies a library of functions--the syntax and semantics of message passing routines--that can be called from other programming languages, 
such as Python and C. MPI provides a very powerful and very general way of expressing parallelism. 
It can be thought of as ``the assembly language of parallel computing,'' because of this 
generality and the detail that it forces the programmer to deal with
\footnote{Parallel Programming with MPI, by Peter S. Pacheco, p. 7}.
In spite of this apparent drawback, MPI is important because it was the first portable, 
universally available standard for programming parallel systems and is the \emph{de facto} standard. 
That is, MPI makes it possible for programmers to develop portable, parallel software libraries,
an extremely valuable contribution.
Until recently, one of the biggest problems in parallel computing was the lack of software. 
However, parallel software is growing faster thanks in large part to this standardization. 

\begin{problem}
Most modern personal computers now have multicore processors. 
In order to take advantage of the extra available computational power, a single program must be specially designed. 
Programs that are designed for these multicore processors are also ``parallel'' programs, typically written using POSIX threads or OpenMP.
MPI, on the other hand, is designed with a different kind of architecture in mind. 
How does the architecture of a system for which MPI is designed differ what POSIX threads or OpenMP is designed for? 
What is the difference between MPI and OpenMP or Pthreads?
\end{problem}

\section*{Why MPI for Python?}
In general, parallel programming is much more difficult and complex than in serial. 
Python is an excellent language for algorithm design and for solving problems that 
don't require maximum performance. 
This makes Python great for prototyping and writing small to medium sized parallel programs. 
This is especially useful in parallel computing, where the code becomes especially complex. 
However, Python is not designed specifically for high performance computing and its 
parallel capabilities are still somewhat underdeveloped, so in practice it is better 
to write production code in fast, compiled languages, such as C or Fortran.

We use a Python library, mpi4py, because it retains most of the functionality of 
C implementations of MPI, making it a good learning tool since it will be easy to 
translate these programs to C later. 
There are three main differences to keep in mind between mpi4py and MPI in C:
\begin{itemize}
    \item Python is array-based. C and Fortran are not.
    \item mpi4py is object oriented. MPI in C is not.
    \item mpi4py supports two methods of communication to implement each of the basic MPI commands. 
    They are the upper and lower case commands (e.g. \li{Bcast(...)} and \li{bcast(...)}). 
    The uppercase implementations use traditional MPI datatypes while the lower case use 
    Python's pickling method. Pickling offers extra convenience to using mpi4py, 
    but the traditional method is faster. In these labs, we will only use the uppercase functions.
\end{itemize}


\section*{Introduction to MPI}
As tradition has it, we will start with a Hello World program.
\lstinputlisting[style=fromfile]{hello.py}
Save this program as \texttt{hello.py} and execute it from the command line as follows:
\begin{lstlisting}[style=ShellInput]
$ mpirun -n 5 python hello.py
\end{lstlisting}
The program should output something like this:
\begin{lstlisting}[style=ShellOutput]
Hello world! I'm process number 3.
Hello world! I'm process number 2.
Hello world! I'm process number 0.
Hello world! I'm process number 4.
Hello world! I'm process number 1.
\end{lstlisting}
Notice that when you try this on your own, the lines will not necessarily print in order. 
This is because there will be five separate processes running autonomously, and we cannot 
know beforehand which one will execute its \li{print} statement first.

\begin{warn}
It is usually bad practice to perform I/O (e.g., call \li{print}) from any process 
besides the root process, though it can oftentimes be a useful tool for debugging.
\end{warn}

\section*{Execution}
How does this program work? As mentioned above, mpi4py programs are follow the single-program 
multiple-data paradigm, and therefore each process will run the same code a bit differently. 
When we execute 
\begin{lstlisting}[style=ShellInput]
$ mpirun -n 5 python hello.py
\end{lstlisting}
a number of things happen:

First, the mpirun program is launched. 
This is the program which starts MPI, a wrapper around whatever program you to pass into it. 
The ``-n 5'' option specifies the desired number of processes. 
In our case, 5 processes are run, with each one being an instance of the program ``python''. 
To each of the 5 instances of python, we pass the argument ``hello.py'' which is the name of our program's text file, 
located in the current directory. Each of the five instances of python then opens the 
\texttt{hello.py} file and runs the same program. 
The difference in each process's execution environment is that the processes are given 
different ranks in the communicator. Because of this, each process prints a different 
number when it executes.

MPI and Python combine to make wonderfully succinct source code. 
In the above program, the line \li{from mpi4py import MPI} loads the MPI module from the mpi4py package.
The line \li{COMM = MPI.COMM_WORLD} accesses a static communicator object, which represents a group of processes which can communicate with each other via MPI commands.

The next line, \li{RANK = COMM.Get_rank()}, is where things get interesting. 
A ``rank'' is the process's id within a communicator, and they are essential to learning 
about other processes. When the program mpirun is first executed, it creates a 
global communicator and stores it in the variable \li{MPI.COMM_WORLD}. 
One of the main purposes of this communicator is to give each of the five processes a 
unique identifier, or ``rank''. When each process calls \li{COMM.Get_rank()}, 
the communicator returns the rank of that process. 
\li{RANK} points to a local variable, which is unique for every calling process 
because each process has its own separate copy of local variables. 
This gives us a way to distinguish different processes while writing all of the source code 
for the five processes in a single file.

In more advanced MPI programs, you can create your own communicators, which means that 
any process can be part of more than one communicator at any given time. 
In this case, the process will likely have a different rank within each communicator.


Here is the syntax for \li{Get_size()} and \li{Get_rank()}, where \li{Comm} is a communicator object:
\begin{description}
\item[Comm.Get\_size()]
Returns the number of processes in the communicator. It will return the same number to every process.
Parameters:
\begin{description}
    \item[Return value] - the number of processes in the communicator
    \item[Return type] - integer
\end{description}
Example:
\lstinputlisting[style=FromFile]{Get_size_example.py}
\item[Comm.Get\_rank()]
Determines the rank of the calling process in the communicator.
Parameters:
\begin{description}
    \item[Return value] - rank of the calling process in the communicator
    \item[Return type] - integer
\end{description}
Example:
\lstinputlisting[style=FromFile]{Get_rank_example.py}
\end{description}

\begin{problem}
Write the ``Hello World'' program from above so that every process prints out its 
rank and the size of the communicator 
(for example, process 3 on a communicator of size 5 prints ``Hello World from process 3 out of 5!'').
\end{problem}

\section*{The Communicator}
A communicator is a logical unit that defines which processes are allowed to 
send and receive messages. In most of our programs we will only deal with the \li{MPI.COMM_WORLD} 
communicator, which contains all of the running processes. 
In more advanced MPI programs, you can create custom communicators to group only a small 
subset of the processes together. By organizing processes this way, MPI can 
physically rearrange which processes are assigned to which CPUs and optimize your 
program for speed. Note that within two different communicators, the same process 
will most likely have a different rank.

Note that one of the main differences between mpi4py and MPI in C or Fortran, 
besides being array-based, is that mpi4py is largely object oriented. 
Because of this, there are some minor changes between the mpi4py implementation of 
MPI and the official MPI specification.

For instance, the MPI Communicator in mpi4py is a Python class and MPI functions 
like \li{Get_size()} or \li{Get_rank()} are instance methods of the communicator class. 
Throughout these MPI labs, you will see functions like \li{Get_rank()} presented as 
\li{Comm.Get_rank()} where it is implied that \li{Comm} is a communicator object.

\section*{Separate Codes in One File}
When an MPI program is run, each process receives the same code. 
However, each process is assigned a different rank, allowing us to specify separate 
behaviors for each process. In the following code, all processes are given the same two numbers. 
However, though there is only one file, 3 processes are given completely different instructions 
for what to do with them. Process 0 sums them, process 1 multiplies them, 
and process 2 takes the maximum of them:
\lstinputlisting[style=FromFile]{separateCode.py}

\begin{problem}
Write a program in which the the processes with an even rank print ``Hello'' and 
process with an odd rank print ``Goodbye.'' 
Print the process number along with the ``Hello'' or ``Goodbye'' 
(for example, ``Goodbye from process 3'').
\end{problem}

\begin{problem}
Sometimes the program you write can only run correctly if it has a certain number of processes. 
Although you typically want to avoid writing these kinds of programs, sometimes it is 
inconvenient or unavoidable. Write a program that runs only if it has 5 processes. 
Upon failure, the root node should print ``Error: This program must run with 5 processes'' 
and upon success the root node should print ``Success!'' To exit, call the function 
\li{COMM.Abort()}.
\end{problem}
